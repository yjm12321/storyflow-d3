\documentclass{report}
\usepackage{graphicx}
\usepackage{hyperref}
\usepackage{url}
\usepackage[left=25mm,right=25mm,top=25mm]{geometry}


\begin{document}

\title{Storyflow - Tracking the Evolution of Stories\\
Information Visualization, SS 2015}
\author{Andreea Muscalagiu 1456893\\
Sorin Adrian Robert Davidoi		1456867}
\maketitle
\chapter{Problem definition}
\par
A narrative or story is any report of connected events, actual or imaginary, presented in a sequence of written or spoken words, or still or moving images \cite{Narrative}. Every story has entities who interact with each other and locations where events take place. The interaction of multiple entities during two adjacent time frames is a session. In a storyline visualization, these entities are represented as lines and relationships between them are mapped to the distance between their corresponding lines. Understanding how relationships between characters evolve over the course of the story is an interesting subject for various applications. Such applications could be analyzing tweets related to a particular story or the plot of a movie. Since the data set we have chosen represents the plot of a movie, we call the entities characters.

\par The paper StoryFlow - Tracking the Evolution of Stories by Liu et al. deals with creating visualizations that help users better analyze a story, particularly the connections between characters and locations. \cite{liu} By considering algorithms and techniques included in the paper, our task was to provide a visualization that facilitates this analysis. The result should be aesthetically pleasing, easy to follow, interactive and it should support hierarchical locations and the rendering of a large number of characters. This can be achieved by minimizing 4 metrics: line crossings, line wiggles, wiggle distance and white space. Line crossings create visual clutter and may lead to occlusion. Line wiggles represent how straight the lines are. The reason to minimize them is that wiggled lines are harder to follow and increase the visual clutter. Wiggle distance is related to line wiggles. Minimizing it leads to a more compact layout. The last metric, white space, is minimized in order not to waste screen space or have an unbalanced layout.

\par
The optimization problem is split into two parts: discrete optimization to minimize the first 2 metrics and continuous optimization to minimize the last 2 metrics. Minimizing the number of line crossings affects the visualization mostly, so it is the first step in the algorithm and it consists of ordering lines, sessions and locations such that there is minimum occlusion. The second step deals with aligning as many sessions and lines as possible between adjacent time frames. Finally, to generate a compact and pleasing storyline visualization, the wiggle distance and white space must be optimized. A set of interactions are also provided, according to the paper, such as adding, deleting, repositioning and straightening of lines, as well as bundling of sessions. To sum up, our tasks were the following:
\begin{itemize}
\item choose a data set to visualize
\item minimize line crossings
\item minimize line wiggles
\item minimize white space and wiggle distance
\item add real-time interaction
\end{itemize}

\chapter{Our Approach}
\par
We have decided to implement a part of the algorithms for optimizing the storyline visualization and also add some user interaction, such as highlighting one or more characters on click, choosing which characters to include in the visualization, choosing a data set and selecting the interpolation function. The algorithm we have chosen to implement is the ordering algorithm, which involves minimizing line crossings by sorting the locations, the sessions and the characters. The main goal was to obtain a good initial layout. We did this by reading a reference paper \cite{4308636} where a graph sweeping algorithm is described and by extending this technique, as mentioned in the StoryFlow paper. It involves computing the ordering of every time frame according to the previous time frame, which is considered fixed. After that, the time frames are traversed again to set the order of the characters as it was computed in the last time frame. These two steps are done two times: traversing the frames from left to right and from right to left. This sweeping is done for a maximum number of iterations which we have chosen. At every step, the barycenter method presented in the referenced paper is used for sorting the characters and the sessions.
\par
We have followed the rules regarding how to represent characters and their relationships. Additionally, we have added a representation for locations as colored closed contours around the characters in that location. We have used the data sets provided on Tanahashi's website \cite{website:tanahashi} and transformed them into the JSON format for easier processing in JavaScript.

\chapter{Related Work}
\centerline{\includegraphics{movie_narrative_chart_star_wars_resized}}
\par
\begin{center}
Star Wars visualization by R. Munroe
\end{center}
\par
Such a storyline visualization was first introduced by R. Munroe through his movie narrative chart \cite{xkcd}. The lines of the characters run from left to right and their relationships are mapped to the distance between their lines at every time frame. Lines are adjacent when characters are in the same location. This visualization was drawn by hand and has lead to research on automatically generating a storyline layout.
\par
An existing visualization technique, designed by Tanahashi and Ma \cite{tanahashi} produces a pleasant-looking storyline visualization. Unfortunately, it takes too much time, when having a large number of characters, because it is based on a genetic algorithm. Therefore, no user interaction is provided. The implementation in Python and the data sets are available on Tanahashi's website \cite{website:tanahashi}. The paper on which our project is based compares its results to Tanahashi's results and uses the same data sets. Its results are considerably better. There is also an online editing tool called PlotWeaver \cite{plotweaver}, which allows the user to create a storyline visualization.

\chapter{Implementation}
\par
In our implementation we have decided to use JavaScript and create the visualizations using D3. The two important parts of our program are the algorithm that orders the locations, sessions and characters and the actual visualization. In the function buildDrawStoryFlow, the two important functions are called, which are annotateDataset and drawStoryFlow, the first one being responsible with the algorithm, and the second with the drawing of the storyline.

\section{Visualization}
\par
For our visualization, we have a few options that we can change, such as the interpolation function for drawing the lines, how thick the lines should be by default, how thick they should be when highlighted, the duration of transition from viewing all the lines to highlighting one or more particular lines, the duration of transition from one data set to another, how large should the padding of the locations be, the font size of the location names (to distinguish them from character names), split length between time frames inside the same location to decide when to split the location into multiple polygons, which characters are chosen to be visualized, opacity of non-highlighted lines and maximum number of iterations, which is strictly concerned with the algorithm. Only toKeep and interpolation are affected by the user interaction and can be changed by the user.
\par
In the function drawStoryFlow we set attributes necessary for visualizing the character lines, such as color and coordinates, as well as the colors of the location contours. We need to scale our time frame to the screen coordinates and compute the path of each character.
In order to visualize the locations, we need to give some points to the function d3.svg.area around which to draw the closed contours. For every location, for every time frame, we select the highest point (with the lowest y coordinate) and the lowest point (with the highest y coordinate) and we set them as top, respectively as bottom. We remember these points, which we give to the area function, while including the location padding in order to make sure the contour is around the lines, not on the character lines themselves. But before this, for the locations inside which there is a big time break between two consecutive time frames, in order to free the space on the screen, we split the contour of one location into multiple contours. The name of the location is written above every of its contours. The colors for the locations have been previously chosen.
\par
Because one might want to follow the journey of one or more particular characters in order to analyze their relationship, we have implemented character highlighting. When you click on one or more characters, the others fade away and the lines of the selected ones are thickened. They can be deselected by clicking on the background area. This is done using transitions, which switch from one stroke width to another, respectively from one opacity to a different one. The names of the characters and the locations need to be bound to their corresponding lines or paths.

\section{Ordering algorithm}
\par
Because we found the algorithms presented in the paper quite complex, we have focused only on minimizing the most important metric, which is represented by the line crossings. This is done in the function annotateDataset using an ordering algorithm in 2 steps: ordering the locations and ordering the sessions and characters.
\par For each time frame, we compute a hierarchical tree, which encodes the locations, sessions and characters that belong to that time frame (the locations are at the first level of the tree, sessions on the second and the characters on the third).
\par
In the function optimizeMoment, we optimize a hierarchical tree by sorting the locations decreasingly by the number of characters that are in the location in that time frame. The sessions and characters are sorted accorting to their weights (which are computed elsewhere and passed as parameters).
\par
\section{Limitations}
\centerline{\includegraphics[width=\textwidth]{iterations_3}}
\par
\begin{center}
Star Wars with 3 iterations
\end{center}
\centerline{\includegraphics[width=\textwidth]{iterations_300}}
\par
\begin{center}
Star Wars with 300 iterations
\end{center}

In general, the current implementation seems to lead to a pretty good looking layout, but there are moments when the algorithm seems to rather complicate the visualization than simplify it. This means that improvements can be made to the implementation. The resulting visualization depends on the maximum number of iterations, which is set to 3 by default. If the user is willing to wait for a longer time, a smaller number of crossings can be obtained, thus, usually, a better layout. One problem that we have noticed is that, when changing the data set, for a while, the highlighting of the characters does not seem to work. We did not manage to figure out why that happens, but somehow the click events are not attached to the character lines. Still, it does not happen when changing the interpolation function.
Optimizations can also be made regarding the position of the location labels (sometimes, depending on the interpolation and path of the characters, it can be hard to read). Unfortunately, we did not have the time to implement a solution which displays the labels of the characters regardless of how the user interacts with the visualization (if the user scrolls to the right, the labels are not visible and it is hard to tell which character belongs to which path).
Also, we did not handle the case of hierarchical locations.

\chapter{Conclusion}
\par
\centerline{\includegraphics[width=\textwidth]{jurassic_park}}
\par
\begin{center}
Jurassic Park, StoryFlow
\end{center}
\centerline{\includegraphics[width=\textwidth]{after_overview}}
\par
\begin{center}
Star Wars, our implementation
\end{center}

In conclusion, we consider that creating visualizations using D3 is a satisfying activity, given that it takes little effort to obtain considerable results, while the algorithmic part is rather frustrating because of the difficulty with manipulating matrices in JavaScript. Still, we would like to try to implement the rest of the optimization techniques in order to see how they improve the layout of our visualization. Another improvement we could make is to split the HTML and JavaScript code into separate files. We would like to add that we found it difficult to understand a scientific paper, and especially reproduce the steps described in it, as the authors are not always specific in their explanations. It would've been very useful to have an open source version of the algorithm in order to compare the solutions and make sure that we understood the steps to be followed.

\bibliographystyle{plain}  
\bibliography{bibliography}

\end{document}